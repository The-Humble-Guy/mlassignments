\documentclass[a4paper, 12pt]{article}
\usepackage{bmstu-title-new}
\usepackage[export]{adjustbox}
\usepackage{pdfpages}

\setcounter{tocdepth}{5}
\setcounter{secnumdepth}{4}

\worknumber{1}
% \variant{3}
\workname{Дескриптивный анализ данных}
\discipline{Методы машинного обучения}
\group{ИУ6-21М}
\author{А. А. Куценко}
\tutor[Преподаватель]{С. Ю. Папулин}
\inspector[Нормоконтролер]{О. Ю. Ерёмин}

\input{src/vars}

\hypersetup{
    pdftitle={Отчет по ДЗ1 Методы машинного обучения Куценко А ИУ6-21М},
}

\begin{document}

	\bmstutitlehome

	\includepdf[pages=-]{\notebookpdf}

\end{document}


% Примеры того, как вставить картинку, таблицу, листинг

	\begin{figure}[H]
		\centering
		\includegraphics[height=0.4\textheigh]{images/r2.png}
		\caption{\centering Результат рbаботы при достижении правила $R_2$ (после $R_2$ программа сделала попытку поставить диагноз правилами $R_3$ и $R_4$, но ни одно из них не сработало)}
		\label{fig:r2}
	\end{figure}

	\begin{table}[H]
		\caption{Варианты заданий}
		\label{tab:lab_variant}
		\begin{tabular}{|p{0.13\textwidth}|p{0.82\textwidth}|}
			\hline
			\textbf{Вариант}   & \textbf{Задание} \\ \hline
			Вариант 3          & \scriptsize Основными симптомами язвы являются стойкие боли.
			                    Пациент ощущает их долго, в зависимости от своей терпеливости~--- неделю, месяц, полгода.
			                    Язвенная боль чаще локализуется в подложечной области, на середине расстояния между пупком и концом грудины;
			                    при язве желудка~--- по средней линии или слева от нее;
			                    при язве двенадцатиперстной кишки~--- на 1\,--\,2\,см вправо от средней линии.

			                    Для язв, протекающих с чрезмерно высокой кислотностьюю, характерны запоры, нередко с кишечными коликами.
			                    Наконец, у язвенных больных часто наблюдается чувство внутренней напряженности и повышенная раздражительность. \\ \hline
		\end{tabular}
	\end{table}

	\begin{singlespacing}
		\small
		\captionsetup{labelsep=endash, justification=raggedright, singlelinecheck=off}
		\lstinputlisting[label=code:data_base, caption={Содержимое базы фактов предметной области для варианта 3}]{code/dbfile.txt}
	\end{singlespacing}

